\pagenumbering{arabic}
\setcounter{page}{1}
\section{Einleitung}

\begin{wrapfigure}[12]{r}[0cm]{180px}
	\vspace{-12px}
	\centering
	\includegraphics[width=1\linewidth]{../graphic/images/firma/Firmengebaude}
	\caption{Firmengbäude}
	\label{fig:firmengebaude}
\end{wrapfigure}

Die folgende Projektdokumentation beschreibt den Ablauf des IHK-Abschlussprojektes, welche der Autor im Rahmen der Ausbildung zum Fachinformatiker Fachrichtung Anwendungsentwicklung durchgeführt hat. Ausbildungsbetrieb ist die ADITO Software GmbH, ein Hersteller für hochflexible Business-, CRM- und xRM-Software mit Sitz in Geisenhausen. Das inhabergeführte Unternehmen bietet Entwicklung, Vertrieb, Projektierung und Service aus einer Hand. Kunden von ADITO kommen aus den unterschiedlichsten Branchen. Neben namhaften mittelständischen Unternehmen, zum Beispiel Ravensburger, Erlus oder Birco, gehören auch große Organisationen wie die WWK Versicherungsgruppe oder die Bundesagentur für Arbeit zu ihren Referenzen.

Die Mitarbeiterzahl des Unternehmens ist insbesondere in den letzten Jahren stark gewachsen. Im Jahr 2018 hat ADITO die Grenze von 100 Mitarbeitern überschritten. Um dafür genügend Raum zu bieten, wurde im gleichen Jahr ein neues Firmengebäude (\autoref{fig:firmengebaude}) errichtet.

\subsection{Beschreibung}
 
In der neuesten Version des xRM-Systems ADITO5 wurde eine neue Clientvariante entwickelt. Nun ist es möglich neben dem konventionellen Java Swing Client, einen Webclient bzw. Browserclient einzusetzen. Dieser bietet einige Vorteile. Beispielsweise ist keine Installation notwendig und die Nutzung ist auf allen Geräten mit Webbrowsern (PC, Tablet und Smartphone) möglich. 

Mit einem Webclient gehen allerdings auch einige Herausforderungen einher. So auch bei der Vergabe von Tastaturkürzeln. Browser behalten es sich vor, einige Shortcuts für eigene Aktionen zu reservieren und so nicht für die eigentliche Webanwendung zur Verfügung zu stellen. Beispiele für solche Shortcuts sind \emph{Strg + P} für Drucken oder \emph{Strg + F} für Suchen. Der Überblick über die Verwendbarkeit von Tastaturkürzeln geht schnell verloren, da diese in jeder Browser-Betriebssystem-Permutation variieren kann.

Im Rahmen dieser Arbeit soll eine Möglichkeit geschaffen werden, bei der Vergabe von Shortcuts innerhalb der hauseigenen xRM-Entwicklungsumgebung (ADITO5 Designer) Unterstützung zu bieten.

\subsection{Ziel}

Um den Entwicklern unserer xRM-Software die Vergabe von Shortcuts zu erleichtern, soll ein spezieller Shortcut-Editor erstellt werden. Dieser muss die Eingabe und Bearbeitung von Shortcuts ermöglichen und bei der Wahl des passenden Tastenkürzels zuarbeiten.

Warnungen im Editor verdeutlichen, dass der eingegebene Shortcut zu Problemen auf einem bestimmten Browser führen kann. Damit der Entwickler feststellen kann, warum ein Shortcut problematisch ist, sollen weitere Informationen angezeigt werden. Diese können beispielsweise angeben, bei welchem Browser bzw. welcher Version das Tastaturkürzel bereits verwendet wird.

\subsection{Umfeld}

Durchgeführt wird das Projekt in der Entwicklungsabteilung, welche auch für die Umsetzung von ADITO5 zuständig war. Die Notwendigkeit des Editors wurde im Zuge der Weiterentwicklung festgestellt. Dadurch kann man die Abteilung Entwicklung selbst als Auftraggeber ihres eigenen Projekts ansehen.

Die Implementierung des Editors wird in der objektorientierten Programmiersprache Java und mithilfe der Entwicklungsumgebung IntelliJ IDEA durchgeführt. Als Framework für die grafische Benutzeroberfläche dient das Java-Swing-Framework.

\subsection{Begründung}

Da gewährleistet werden soll, dass vergebene Tastenkürzel auf allen relevanten Browsern funktionieren, muss dem Entwickler bei der Wahl eines passenden Shortcuts immer klar sein, ob dieser von den entsprechenden Browsern unterstützt wird. Da jeder Browser andere Shortcuts vorbelegt und diese sich je nach Betriebssystem wieder unterscheiden können, ist eine manuelle Überprüfung durch den Projektierer unzumutbar. So müsste dieser auf jedem Betriebssystem alle relevanten Browser testen. Ein derartig enormer Aufwand und die mögliche Fehlvergabe von Tastenkombinationen können durch die technische Assistenz mittels des genannten Editors vermieden werden.

\subsection{Schnittstellen}
\label{schnittstellen} 

Um herauszufinden, welche Shortcuts die verschiedenen Browser auf unterschiedlichen Betriebssystemen verwenden, wurde außerhalb dieser Abschlussarbeit eine Testanwendung implementiert. Diese kann auf verschiedenen Betriebssystemen (z. B. Windows, Linux und MacOS) ausgeführt werden und testet alle möglichen Shortcut-Kombinationen für die verbreitetsten Browser (z. B. Chrome, Firefox oder Safari). Das Ergebnis dieser Tests wird in Form von XML-Dateien gespeichert (Beispiel siehe Anhang \ref{xml}). Für jede Browser-Betriebssystem-Kombination existiert eine eigene Datei, in welcher alle problembehafteten Shortcuts verzeichnet sind.

Damit die in den Dateien enthaltenen Informationen dem Benutzer dargestellt werden können, muss der Editor das Einlesen und Verarbeiten von XML-Strukturen beherrschen. Hierfür kommt das hauseigene \emph{Propertly} Framework zum Einsatz. Dieses kümmert sich um sämtliche XML-spezifische Arbeiten und ermöglicht so eine komfortable Nutzung.

Die über den Editor ausgewählten Shortcuts werden Aktionen zugeordnet, welche bereits im ADITO5 Designer implementiert wurden. Zudem existieren sogenannte Entitys, welche Dateneinheiten darstellen und eben genannte Aktionen besitzen können. Beispielsweise könnte ein Entity \glqq Firma\grqq\xspace existieren, welches wiederum die Aktion \glqq Mitarbeiter hinzufügen\grqq\xspace besitzt. Dieser Aktion könnte nun ein Shortcut zugewiesen werden z. B. \emph{Strg + Einfg}.


\subsection{Abgrenzung}

Aufgrund des beschränkten Projektumfangs ist das Einbinden des Editors in den bestehenden ADITO5 Designer nicht Bestandteil der Projektarbeit.






\newpage