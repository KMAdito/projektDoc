\section{Fazit}
\par Der produktive Einsatz des Remote-Loggers wird weitere Anforderungen der Administratoren und ADITO4-Projektentwickler aufzeigen. Es wurde hierdurch eine Möglichkeit geschaffen, direkt auf die Ausgaben des ADITO4-Servers zuzugreifen. Das hat den Vorteil, dass nun nicht mehr per Fernzugriff auf das Hostsystem der ADITO4-Kundenserver verbunden werden muss, um dessen Meldungen zu lesen. \\
Der Remote-Logger bietet auch im Vergleich zum bisherigen \glqq FileLogger\grqq\ den entscheidenden Echtzeit-Vorteil, denn der ADITO4-FileLogger schreibt alle erhaltenen CheckPoints blockweise in seine Datei. Somit werden Ausgaben verzögert geschrieben und es kann erst verspätet auf diese reagiert werden.

\par Es ist denkbar, dass das Feature des Remote-Loggers noch mit einer Exportfunktion erweitert wird. Somit könnte man Log-Dateien erstellen, die man wiederum mit dem \glqq LogFileViewer\grqq\ des ADITO4-Managers betrachten kann. \\
Ebenso wäre es möglich einen Filter zu implementieren, der alle Nachrichten die der Benutzer nicht sehen möchte, herausfiltert. Beispielweise werden dann nur noch Nachrichten mit der Priorität \glqq hoch\grqq\ angezeigt. Einstellbar soll dies mit verschiedenen Buttons und Eingabefelder werden. Ein Filter nach angemeldeten Benutzern ist von der ADITO-Geschäftsleitung ebenfalls gewünscht, denn somit könnten auftretende Fehler am ADITO4-Client leichter identifiziert und behoben werden. \\
Eine zusätzliche Erweiterung des Remote-Loggers könnte die Verschlüsselung des Datenaustausches zwischen Remote-Logger-Server und Remote-Logger-Client sein. Dann könnte nahezu komplett ausgeschlossen werden, dass unberechtigte Dritte Zugriff zu den vom Remote-Logger-Server gesendeten Daten erhalten. Hierzu käme SSL in Frage. SSL wurde bereits bei der Kommunikation zwischen ADITO4-Server und ADITO4-Client verwendet, was ein Wiederverwenden von bereits bestehendem Code erlaubt.