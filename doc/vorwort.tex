\pagenumbering{arabic}
\setcounter{page}{1}
\section{Vorwort}
\subsection{ADITO Software GmbH}
\begin{wrapfigure}[8]{r}[0cm]{180px}
	\vspace{-8px}
	\centering
	\includegraphics[width=175px]{../img/Firmengebaeude}
	\caption{Firmengebäude}
\end{wrapfigure}

Die ADITO Software GmbH zählt zu den führenden Herstellern hochflexibler Business-, CRM- und xRM-Software. ADITO bietet Entwicklung, Vertrieb, Projektierung und Service aus einer Hand. Das inhabergeführte Unternehmen legt großen Wert auf die persönliche Beratung und Kundenbetreuung durch ihre exzellent ausgebildeten Mitarbeiter.

Kunden von ADITO kommen aus den unterschiedlichsten Branchen. Neben namhaften mittelständischen Unternehmen, zum Beispiel Ravensburger, Birco oder Erlus, gehören auch große Organisationen wie die WWK Versicherungsgruppe, IG Metall, Kassenärztliche Vereinigungen oder die Bundesagentur für Arbeit zu unseren Referenzen. Insgesamt setzen rund 800 Kunden auf Service, Innovationsstärke und Kontinuität von ADITO.

\subsection{Intention}
\par 
In der neuesten Version des xRM-Systems ADITO5 wurde eine neue Clientvariante entwickelt. Nun ist es möglich neben dem konventionellen Java Swing Client einen Webclient bzw. Browserclient einzusetzen. Dieser bietet einige Vorteile. Beispielsweise ist keine Installation mehr notwendig und er kann auf allen Geräten (PC, Tablet und Smartphone) genutzt werden. Allerdings gehen damit auch einige Herrausforderungen einher. So auch bei der Vergabe von Tastaturkürzeln. Browser behalten es sich vor, einige Shortcuts für eigene Aktionen zu reservieren und so nicht für die eigendliche Webanwendung zur Verfügung zu stellen. Beispiele für solche Shortcuts währen Strg + P für Drucken oder Strg + F für Suchen. Der Überblick über die Verwendbarkeit von Tastaturkürzeln geht schnell verloren, da diese in jeder Browser Betriebssystem permutation variieren kann.

Um den Projektierern unserer xRM-Software die Vergabe von Shortcuts zu erleichtern soll ein spezieller Shortcut-Editor entwickelt werden, der die Eingabe und Bearbeitung von Shortcuts per Tastatur und Maus ermöglicht und bei der Wahl des passenden Shortcuts Unterstützung bietet.
Dafür soll bei Tastaturkürzeln gewahrnt werden, wenn sie auf einem Browser Probleme breiten könnten.
Um feststellen zu können warum ein Shortcut problematisch ist, sollen weitere Informationen angezeigt werden, welche beispielsweise angeben, in welchem Browser bzw. welcher Version das Tastaturkürzel bereits verwendet wird.
\par
\newpage