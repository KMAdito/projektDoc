\section{Testphase}

Um sicherzustellen, dass alle implementierten Komponenten anforderungsgerecht funktionieren, wurde der Editor anhand manueller Akzeptanztests getestet. Dabei wurde darauf geachtet, auf jede mögliche Bedienweise einzugehen. So wurde der Editor sowohl per Maus als auch per Tastatur bedient. Gleichzeitig wurde stets überprüft, ob sich der Editor wie gewünscht verhält. Konnte man dabei Unstimmigkeiten feststellen, so wurden die jeweiligen Fehler im Sourcecode behoben und erneut getestet.

Damit die grundlegende Funktionalität auch dauerhaft gewährleistet werden kann, wurde zudem ein automatisierter Test entwickelt (siehe Anhang \ref{Test_ShortcutEditor}). Dabei wurde das Test-Framework \emph{AssertJ Swing} in Kombination mit \emph{JUnit} verwendet.

Das Framework \emph{AssertJ Swing} ist eine Weiterentwicklung des älteren open-source Frameworks \emph{FEST}. Es ermöglicht die einfache Simulation eines Benutzers für Swing Applikationen. Da es als fluent-Api entwickelt wurde, ist der Code sehr gut lesbar und einfach zu verstehen.

Bei \emph{JUnit} handelt es sich ebenfalls um ein quelloffenes Framework, welches durch ein Maven-Plugin in den Build-Vorgang von ADITO eigebunden ist. Dadurch werden alle vorhandenen Tests bei jedem Kompiliervorgang mit Maven automatisch ausgeführt. So können etwaige Fehler schon während der Entwicklung erkannt und vermieden werden.

\begin{wrapfigure}[9]{r}[0px]{322px}
    \centering
	\vspace{-12px}
	\begin{spacing}{0.75}
		\begin{javacode}[firstnumber=27]
@Before
public void setUp()
{
  _setLookAndFeel();
  
  dummyStore = new ShortcutDummyStore();
  ui = GuiActionRunner.execute(ShortcutEditorUI::new);
  Dialog shortcutDialog = GuiActionRunner.execute(
      () -> ShortcutEditorDialog.createDialog(dummyStore, ui));
  
  dialog = new DialogFixture(shortcutDialog);
  dialog.show();
}\end{javacode}
	\end{spacing}
	\caption{Setup-Methode}
	\label{fig:Test-ShortcutEditor-onSetUp}
\end{wrapfigure}

In \autoref{fig:Test-ShortcutEditor-onSetUp} ist der Sourcode der Initialisieungsmethode \emph{onSetUp()} dargestellt. Um JUnit anzuweisen, die Methode vor jedem Teststart auszuführen, wurde sie mit der Annotaion \emph{@Before} versehen. 

\vspace{10px}

Nach dem Setzen des ADITO spezifischen Look and Feels, wird der DummyStore und die UI initialisiert. Zum Erstellen des Dialogs werden diese Bestandteile der Methode \emph{ShortcutEditorDialog.createDialog(...)} übergeben. Damit die Ausführung innerhalb des Event Dispatch Threads von Swing erfolgt, kommt ein \emph{GuiActionRunner} zum Einsatz. Dieser führt eine beliebige Anweisung im EDT aus, und liefert das Ergebnis zurück. Nachdem der Dialog erzeugt wurde, wird ein \emph{DialogFixture} erstellt. Hierüber lässt sich der Editor automatisiert bedienen und kann so getestet werden.

\begin{wrapfigure}[11]{l}[0px]{322px}
    \centering
	\vspace{-12px}
	\begin{spacing}{0.75}
		\begin{javacode}[firstnumber=28]
@Test
public void test_EnterShortcut()
{
  int i = 0;
  
  for (IEditorShortcutModel model : dummyStore.getAllModels())
  {
    IShortcut shortcut = new Shortcut(EKeyModifier.SHIFT,
                                      EKey.values()[i++]);
    
    _selectNode(model);
    _enterShortcut(shortcut);
    Assert.assertEquals(model.getShortcut(), shortcut);
  }
}\end{javacode}
	\end{spacing}
	\caption{Setup-Methode}
	\label{fig:Test-ShortcutEditor-EnterShortcut}
\end{wrapfigure}

Die mit \emph{@Test} annotierte Testmethode (\autoref{fig:Test-ShortcutEditor-onSetUp}) überprüft, ob eingegebene Shortcuts auch in das selektierte Model übertragen werden. Hierfür wird jedes verfügbare Shortcut Model selektiert und die Eingabe eines Shortcuts simuliert. Der Test ist dann erfolgreich, wenn der Shortcut jedes Models mit dem jeweils zuvor Eingegebenen übereinstimmt.
\newpage