\begin{wrapfigure}[16]{r}[0px]{322px}
    \centering
	\vspace{-5px}
	\begin{spacing}{0.75}
		\begin{javacode}[firstnumber=62]
public void setCheckItemGroup(ICheckItemGroup pRootGroup)
{
  if(!Objects.equals(pRootGroup, rootGroup))
  {
    rootGroup = pRootGroup;
    
    rootCheckItemContainer.setCheckItems(pRootGroup.getItems());
    
    int dept = _getMaxDept(pRootGroup);
    CheckItemContainer children = rootCheckItemContainer;
    
    removeAll();
    for (int i = 0; i < dept; i++)
    {
      CheckItemViewer viewer = new CheckItemViewer(children);
      children = viewer.getChildItemContainer();
      add(viewer);
    }
  }
}\end{javacode}
	\end{spacing}
	\caption{Setzten der CheckItemGroup im CheckItemAccordion}
	\label{fig:CheckItemAccordion-setCheckItemGroup}
\end{wrapfigure}